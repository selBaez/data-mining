\documentclass[conference]{IEEEtran}

\ifCLASSINFOpdf

\else

\fi


\usepackage{graphicx}
\usepackage{textcomp}
\usepackage{mathtools}
\usepackage[backend=bibtex, style=ieee]{biblatex}
\usepackage{tabularx, caption}
\usepackage{hyperref}

\addbibresource{bibliography}

\begin{document}
 
\title{Data Mining - Flavours of Physics}

\author{\IEEEauthorblockN{Selene Baez Santamaria (2572529) \\
		Andrea Jemmett (2573223) \\
		Dimitris Alivanistos (2578740)}

\IEEEauthorblockA{
Vrije Universiteit Amsterdam \\
Amsterdam, Netherlands}}

\maketitle


\begin{abstract}
%TODO
\end{abstract}


\IEEEpeerreviewmaketitle


\section{Introduction}

The goal of this competition is to find undiscovered phenomena given data from the Large Hadron Collider (LHC). It is a classification task for the variable \textit{signal}, which can have values $0$ or $1$, given the values for other $50$ variables. 

The competition provides a Starter Kit, training and test sets, a specific submission format and a couple check files to evaluate the submitted data. The link for these assets is \href{https://www.kaggle.com/c/flavours-of-physics/data}{here} This competition was closed only five months ago, and it is still relevant in the field.

\subsection{Objectives}

\begin{itemize}
	\item Compare the results from having domain knowledge and not.
	\item Objective 2. %TODO
\end{itemize}
 
\subsection{Research question}
\label{sec:researchQuestions}


\subsection{Hypothesis}
\label{sec:hypothesis}
Comparable accuracy (at least as high as the winner of the Kaggle competition) can be achieved by a system that lacks domain knowledge.

\subsection{Contribution}


\subsection{Organization}
Section \ref{sec:related_work} explores the solutions submitted to Kaggle at the time of competition. Section \ref{sec:dataset} describes the dataset and the competition regulations (such as the agreement measures to be met). Section \ref{sec:model_design} depicts the model we proposed to tackle the classification task. Section \ref{sec:experiment} refers to the experimental set up we use to test our model, while Section \ref{sec:results} shows the results obtained. Finally, Section \ref{sec:conclusion} summarizes our observations.


\section{Related Work}
\label{sec:related_work}
%TODO Describe solution from winner

% Maybe include other physics related competitions?


\section{The Dataset}
\label{sec:dataset}
As said, the dataset was offered on Kaggle for a competition organized by
institutions such as CERN and LHCb. The data have been collected directly at the
LHC during experiments with high energy particles collisions. The dataset
consists of a collection of collision events and their properties. The objective
of the Kaggle competition was to predict whether a $\tau \rightarrow 3\mu$ decay
(the one that identifies a lepton flavour) was present in the collision. From
scientists this phenomenon is supposed \emph{not} to happen, so the goal of the
competition was to discover $\tau \rightarrow 3\mu$ happening more frequently
than scientists currently expect.

\subsection{Training Data}
For training there is a labeled dataset ready to be used to train a classifier.
The label, marked as \texttt{signal} with range in ${0,1}$ where 1 identifies
signal events while 0 represents background events). Signal events have been
simulated while background events come from real data collected by the LHCb
detectors, observing collision of accelerated particles with a specific mass
range in which $\tau \rightarrow 3\mu$ can't happen.

The training dataset is given is CSV format and contains 49 features plus target
label. For a detailed description of the features see Appendix \ref{sec:features}

\subsection{Testing}
For this Kaggle competition the submission procedure is different from the usual
ones. The dataset comes with, besides a test set, an \textit{agreement} and a
\textit{correlation} set. Any submission has to pass the agreement and
correlation checks before being scored on the test set.

\subsubsection{Agreement Test}
\label{sec:agreement}
The agreement dataset contains real and simulated events for a much more known,
observed and understood decay: $Ds \rightarrow \varphi\pi$. The motivation for
this check is that since the training set contains simulated data (for a
phenomenon not well understood), it is possible for the classifier to reach high
performances by picking up features that are not well modeled by the simulation.
The check then requires the classifier not to expose a large discrepancy when
applied to real-world and simulated data.

For this score, we are provided with a dataset on the control channel $Ds
\rightarrow \varphi\pi$ which has the same features as the training set. This
type of decay is not present in the training data. The
\textit{Kolmogorov-Smirnov} test is used to evaluate the differences between
real and simulated data between the classifier distribution on each sample.
The Kolmogorov-Smirnov metric has to be smaller than 0.09 to pass the agreement
check.

\subsubsection{Correlation Test}
\label{sec:correlation}
This test checks whether the classifier is uncorrelated with the $\tau$ mass.
Because mass is a measured quantity, scientists don't trust it when building a
model. Correlation with mass in an artificial signal-like peak or lead to
incorrect estimations of background signals.

The \texttt{mass} column is not included in the test dataset. However, this hidden
mass information is used to perform a \textit{Cramer-von Mises} test,
iteratively comparing two distributions of a) predicted values from submission
for entire dataset and b) predicted values within a certain mass region in
rolling window fashion along the whole mass range. Getting similar distributions
for all mass sub-regions means that the classifier is not correlated with the
mass. The submission must give a Cramer-von Mises value less than 0.002 to pass
the correlation test.

\subsubsection{Test Set}
\label{sec:test-set}
The test set has the same columns that the training set has except for
\texttt{mass}, \texttt{production}, \texttt{min\_ANNmuon} and \texttt{signal}.
The data contained in this dataset consists of:
\begin{enumerate}
	\item simulated signal events for $\tau \rightarrow 3\mu$;
	\item real background events for $\tau \rightarrow 3\mu$;
	\item simulated events for the $Ds \rightarrow \varphi\pi$;
	\item real background events for $Ds \rightarrow \varphi\pi$.
\end{enumerate}
Events related to the control channel are not used for scoring, but by the
agreement check (Section \ref{sec:agreement}). One should treat all samples as
coming from the same collision channel during classification.

\section{Model design}
\label{sec:model_design}
The model consists of 30 neural networks with different topologies.

\subsection{Neural Networks design}
\label{sec:NN_design}
%TODO describe topologies and MOTIVATE THEM!
% Inputs

% Outputs

\subsection{Model Ensemble}



\section{Experimental set up}
\label{sec:experiment}

\subsection{Implementation}
\label{sec:implementation}


\subsection{Training}


\subsection{Testing/Evaluation}



\section{Results}
\label{sec:results}

\section{Conclusion}
\label{sec:conclusion}

\clearpage
\appendix

\section{Features for training}
\label{sec:features}
Follows a list of available features for training:
\begin{itemize}
	\item \texttt{FlightDistance} - distance between $\tau$ and PV (primary vertex, the
	original protons collision point);
	\item \texttt{FlightDistanceError} - error on FlightDistance;
	\item \texttt{mass} - reconstructed $\tau$ candidate invariant mass, which
	is \textit{absent in the test samples};
	\item \texttt{LifeTime} - life time of tau candidate;
	\item \texttt{IP} - Impact Parameter of tau candidate;
	\item \texttt{IPSig} - Significance of Impact Parameter;
	\item \texttt{VertexChi2} - $\chi^2$ of $\tau$ vertex;
	\item \texttt{dira} - cosine of the angle between the $\tau$ momentum and line
	between PV and tau vertex;
	\item \texttt{pt} - transverse momentum of $\tau$;
	\item \texttt{DOCAone} - Distance of Closest Approach between p0 and p1;
	\item \texttt{DOCAtwo} - Distance of Closest Approach between p1 and p2;
	\item \texttt{DOCAthree} - Distance of Closest Approach between p0 and p2;
	\item \texttt{IP\_p0p2} - Impact parameter of the p0 and p2 pair;
	\item \texttt{IP\_p1p2} - Impact parameter of the p1 and p2 pair;
	\item \texttt{isolationa} - track isolation variable;
	\item \texttt{isolationb} - track isolation variable;
	\item \texttt{isolationc} - track isolation variable;
	\item \texttt{isolationd} - track isolation variable;
	\item \texttt{isolatione} - track isolation variable;
	\item \texttt{isolationf} - track isolation variable;
	\item \texttt{iso} - track isolation variable;
	\item \texttt{CDF1} - cone isolation variable;
	\item \texttt{CDF2} - cone isolation variable;
	\item \texttt{CDF3} - cone isolation variable;
	\item \texttt{production} - source of $\tau$ (\textit{absent from test data});
	\item \texttt{ISO\_SumBDT} - track isolation variable;
	\item \texttt{p0\_IsoBDT} - track isolation variable;
	\item \texttt{p1\_IsoBDT} - track isolation variable;
	\item \texttt{p2\_IsoBDT} - track isolation variable;
	\item \texttt{p0\_track\_Chi2Dof} - quality of p0 muon track;
	\item \texttt{p1\_track\_Chi2Dof} - quality of p1 muon track;
	\item \texttt{p2\_track\_Chi2Dof} - quality of p2 muon track;
	\item \texttt{p0\_pt} - transverse momentum of p0 muon;
	\item \texttt{p0\_p} - momentum of p0 muon;
	\item \texttt{p0\_eta} - pseudorapidity of p0 muon;
	\item \texttt{p0\_IP} - Impact parameter of p0 muon;
	\item \texttt{p0\_IPSig} - Impact Parameter Significance of p0 muon;
	\item \texttt{p1\_pt} - transverse momentum of p1 muon;
	\item \texttt{p1\_p} - momentum of p1 muon;
	\item \texttt{p1\_eta} - pseudorapidity of p1 muon;
	\item \texttt{p1\_IP} - Impact parameter of p1 muon;
	\item \texttt{p1\_IPSig} - Impact Parameter Significance of p1 muon;
	\item \texttt{p2\_pt} - transverse momentum of p2 muon;
	\item \texttt{p2\_p} - momentum of p2 muon;
	\item \texttt{p2\_eta} - pseudorapidity of p2 muon;
	\item \texttt{p2\_IP} - Impact parameter of p2 muon;
	\item \texttt{p2\_IPSig} - Impact Parameter Significance of p2 muon;
	\item \texttt{SPDhits} - number of hits in the SPD detector;
	\item \texttt{min\_ANNmuon} - muon identification. LHCb collaboration trains
	Artificial Neural Networks (ANN) from informations from RICH, ECAL,
	HCAL, Muon system to distinguish muons from other particles. This
	variable denotes the minimum of the three muons ANN. This feature
	should not be used for training and is \textit{absent from the test
		sets};
	\item \texttt{signal} - is the target variable to predict.
\end{itemize}

\end{document}